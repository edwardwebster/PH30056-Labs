\documentclass[%
 reprint,
%superscriptaddress,
%groupedaddress,
%unsortedaddress,
%runinaddress,
%frontmatterverbose, 
%preprint,
%preprintnumbers,
%nofootinbib,
%nobibnotes,
%bibnotes,
 amsmath,amssymb,
 aps,
%pra,
%prb,
%rmp,
%prstab,
%prstper,
%floatfix,
]{revtex4-2}

\usepackage{graphicx}% Include figure files
\usepackage{dcolumn}% Align table columns on decimal point
\usepackage{bm}% bold math
\usepackage{hyperref}% add hypertext capabilities
%\usepackage[mathlines]{lineno}% Enable numbering of text and display math
%\linenumbers\relax % Commence numbering lines

%\usepackage[showframe,%Uncomment any one of the following lines to test 
%%scale=0.7, marginratio={1:1, 2:3}, ignoreall,% default settings
%%text={7in,10in},centering,
%%margin=1.5in,
%%total={6.5in,8.75in}, top=1.2in, left=0.9in, includefoot,
%%height=10in,a5paper,hmargin={3cm,0.8in},
%]{geometry}

\begin{document}

\preprint{APS/123-QED}

\title{Computational simulation of diffusion limited aggregation}% Force line breaks with \\
%\thanks{A footnote to the article title}%

\author{23532}
\affiliation{Physics Department, University of Bath.}%Lines break automatically or can be forced with \\
%\author{Second Author}%
 \email{ew639@bath.ac.uk}
%\affiliation{%
% Authors' institution and/or address\\
% This line break forced with \textbackslash\textbackslash
%}%

%\date{\today}% It is always \today, today,
             %  but any date may be explicitly specified
             
\begin{abstract}
Diffusion limited aggregation (DLA) is a process by which a particle diffuses across empty space. When the `walker' particle encounters a `seed' particle it aggregates onto this seed particle - and joins the seed so that particles may stick to it. As more particles are simulated, the small body of aggregated particles grows forming a cluster. Importantly, particles will stick to the cluster and cause it to grow.\\

Raw data and programs used are accessible at \href{http://people.bath.ac.uk/ew639/PR4.html}{http://people.bath.ac.uk/ew639/PR4.html}
\end{abstract}

%\keywords{Suggested keywords}%Use showkeys class option if keyword
                              %display desired
\maketitle

%\tableofcontents

\section{\label{sec:Introduction}Introduction}

Diffusion limited aggregation (DLA) is a process by which a particle diffuses across empty space. When the `walker' particle encounters a `seed' particle it aggregates onto this seed particle - and joins the seed so that particles may stick to it. As more particles are simulated, the small body of aggregated particles grows forming a cluster. Importantly, particles will stick to the cluster and cause it to grow.

The resulting cluster formed is a fractal. These are objects which have a non-integer dimension. Conventionally, we are experienced with objects who's dimension is an integer (e.g a 1D line or a 2D square) where the object scales with $2^d$. In these cases the dimension is easily assessed by considering how each object scales. For example, doubling the length of the sides of a square increases its area by 4 which is $2^2$, thus, the dimension of a square is 2. One could easily compare this to doubling the radius of a sphere which causes the volume to increase by eight, hence $d=3$; or similarly to a line which doubles in length. 

A fractal, instead, scales with a non-integer dimension and the value of $d$ is not found as trivially as in the cases above. One method of calculating the dimension of a fractal is box counting, where the fractal is observed on successively smaller grid's of boxes and the number of boxes containing a fragment of the fractal counted. Alternatively, another method counts the number of points ($N$) of size $a$ required to construct an object which has a radius of $R$. The fractal dimension is defined by
\begin{align}
\label{eq:FractalDimensionRelation}
N = \left ( \frac{R}{A} \right )^d
\end{align}
where $d$ is the fractal dimension. The equivalence of these two techniques to establish the fractal dimension will first be verified. Assuming the two methods yield the same result, box counting will be used as the main technique to measure the fractal dimension.

The process by which the cluster formed by DLA grows will be investigated in many scenarios. Initially, the fractal dimension of the `base' case will be measured - this is the dimension of a cluster which grows when a single particle is simulated at a time. The dimension of a cluster which is grown from many non-interacting particles at a given density will be simulated, with the aim of determining how the change in density of particles affects the growth of the cluster. Finally, the same approach will be taken for many particles, however, the growth of the cluster will be analysed as a function of density for many particles which do interact such that they can not occupy the same position.

%An analysis will also be made for systems which start with multiple seed particles.

The DLA simulation is ran in C++. Originally code from A. Souslov and V. Rimpilainen is used an adapted to run the simulation. A new method is developed which is lighter to run and allows for more flexibility in the simulation.

\subsection{Previous Method}
The original code provided by A. Souslov can be downloaded here:

The DLA system is simulated using C++. The provided simulation creates a seed particle at the origin and then simulates particles one at a time which takes a random step in the $x$ or $y$ direction at each time step. When the particle encounters the seed it ceases to move and appends to the cluster, at which point a new particle is created.

New particles are created at a constant distance from the maximum size of the cluster. Any particle which exceeds a specified distance from the cluster radius is destroyed - as it will take a long time to get back to the cluster - and a new one generated to take its place. These methods are designed to speed the simulation up.

Finally the fractal dimension is calculated from the number of particles forming the cluster. The radius of the cluster is estimated and the fractal dimension found for the cluster from the gradient of the line of best fit connecting $\ln(N)$ against $\ln(R)$. According to equation \ref{eq:FractalDimensionRelation}, these should be related by
\begin{align}
\ln(N)=d \ln(R) + \ln(\beta)
\end{align} 
hence, the gradient should measure the fractal dimension.
\subsection{New Method}
In order to conduct further analysis a new code was created. A grid who's size could be specified was initialised with seed particles also specified (normally the origin was used). Random particles were created and at each time step allowed to take independent steps in $x$ and $y$ (allowing for diagonal steps which previously could not be taken). When a particle collides with the cluster it became inactive and joined the cluster. Notably the following properties of the system could be controlled:
\begin{itemize}
    \item the geometry of the simulation
    \item the seed position's and number of seeds
    \item the probability of walker sticking to the cluster/seed
    \item the interactions of walkers
    \item number (density) of walker particles
\end{itemize}
which were not previously accessible using the method from A. Souslov.


% The \nocite command causes all entries in a bibliography to be printed out
% whether or not they are actually referenced in the text. This is appropriate
% for the sample file to show the different styles of references, but authors
% most likely will not want to use it.
\nocite{*}

\bibliography{references}% Produces the bibliography via BibTeX.

\end{document}
%
% ****** End of file apssamp.tex ******
